% Packages
% sudo tlmgr install silence appendixnumberbeamer fira fontaxes mwe noto csquotes babel helvet
%--- Preamble ---------------------------------------------------------%
% Load LaTeX packages
\documentclass[aspectratio=169]{beamer}                    % supports floating text in any location
\usetheme[darkmode]{pureminimalistic}
%\usetheme[lightmode]{pureminimalistic}
\graphicspath{{logos/}}

\usepackage[utf8]{inputenc}
\usepackage{csquotes,xpatch}% recommended
%\usepackage[english]{babel}
%\usepackage[american]{babel}
\usepackage[brazil]{babel}
\babelprovide[import, main]{portuguese}
\babelprovide[import]{english}
\usepackage{tikz}

% \renewcommand{\pageword}{}
% \renewcommand{\logoheader}{\vspace{1.5em}}

\usepackage[
	%    natbib=true,
	backend=biber,
	%    style=abnt,
	%    style=authoryear-comp,
	%    style=authoryear,
	style=ieee,
	%    style=acm,
	%    style=apalike,
	%    style=siam,
	%    style=ieeetr,
	%    style=plain,
	doi=true,
	eprint=false,
	hyperref=true]
{biblatex}
\addbibresource{demo_bib.bib}

%% this makes it possible to add backup slides, without counting them
\usepackage{appendixnumberbeamer}
\renewcommand{\appendixname}{\texorpdfstring{\translate{appendix}}{appendix}}
% logos

% footer page
\renewcommand{\pageword}{Página}

% Math Font Default (Fira is strange)
\renewcommand\mathfamilydefault{\rmdefault}

% if loaded after begin{document} a warning will appear: "pdfauthor already used"
\title[short title]{This is the normal length of a research paper:
	always longer than you would expect}
\author{Fulano de Tal \\
	\texttt{fulano@uni9.pro.br}}
\institute{Universidade Nove de Julho - UNINOVE}
\date{\today}

\begin{document}
% has to be loaded outside of a frame to work!
\maketitle

% For longer table of contents, I find it cleaner to
% use no footline.
\begin{frame}[plain, noframenumbering]{Outline}
	\tableofcontents
\end{frame}

\section{Aspect ratio}
\begin{frame}[fragile]{Aspect ratio}
	This pdf uses a 16:9 aspect ratio. To utilize
	this version, simply use:
	\begin{verbatim}
    \documentclass[aspectratio=169]{beamer}
    \end{verbatim}
	\vfill
	The default is a 4:3 aspect ratio.
	\begin{verbatim}
    \documentclass{beamer}
  \end{verbatim}
\end{frame}

\section{vfilleditems}
\begin{frame}{Using itemize}
	\begin{itemize}
		\item I like it to have my bullet points
		\item evenly spaced from one another
		\item then few bullet points, are not crammed on
		      the upper part of the slide
		      like it is right now with itemize
	\end{itemize}
\end{frame}

\begin{frame}[fragile]{Using vfilleditems}
	\begin{verbatim}
    Use the provided \vfilleditems environment
    to create nicely spaced bullet points.

    \begin{vfilleditems}
      \item I like it to have my bullet points
      \item evenly spaced from one another
      \item then few bullet points, are not crammed on
      the upper part of the slide
    \end{vfilleditems}
    \end{verbatim}
\end{frame}

\begin{frame}{Using vfilleditems}
	\begin{vfilleditems}
		\item I like it to have my bullet points
		\item evenly spaced from one another
		\item then few bullet points, are not crammed on
		the upper part of the slide
	\end{vfilleditems}
\end{frame}

\begin{frame}{Using vfilleditems}
	\begin{vfilleditems}
		\item Note that the overlay specification
		is a bit different to \emph{itemize}
		\item For grouped overlay specifications, simply add it
		directly after the environment:
		\begin{vfilleditems}
			\item \texttt{\textbackslash{}begin\{vfilleditems\}<+->}
		\end{vfilleditems}
	\end{vfilleditems}
\end{frame}


\section{Fonts}
\begin{frame}[fragile]{Fonts}
	Fonts:

	{\small This is small}

	This is normal size

		{\large This is large}
	\vfill
	Per default the \emph{Fira Font} Package is
	used. The \emph{Noto Font} is also bundled into this
	package.
\end{frame}

\begin{frame}[fragile]{Fonts}
	To use \emph{Noto} instead of \emph{Fira Fonts}
	\begin{verbatim}
    \usetheme[noto]{pureminimalistic}
  \end{verbatim}
	\vfill
	To disable the \emph{Fira Fonts} and use the default font
	\begin{verbatim}
    \usetheme[customfont]{pureminimalistic}
  \end{verbatim}
\end{frame}

\section{Color}
\begin{frame}[fragile]{Color}
	To overwrite the theme color
	\begin{enumerate}
		\item Define a new color
		\item redefine the themes color (before document begins)
	\end{enumerate}
\end{frame}

\begin{frame}[fragile]{Change color example}
	\small
	\begin{verbatim}
  \usetheme{pureminimalistic}
  \definecolor{textcolor}{RGB}{0, 0, 120}
  \definecolor{title}{RGB}{0, 0, 0}
  \definecolor{footercolor}{RGB}{133, 133, 133}
  \definecolor{bg}{RGB}{25, 116, 210}

  \renewcommand{\beamertextcolor}{textcolor}
  \renewcommand{\beamerbgcolor}{bg}
  \renewcommand{\beamerfootertextcolor}{footercolor}
  \renewcommand{\beamertitlecolor}{title}
  \end{verbatim}
\end{frame}

\begin{frame}[fragile]{Dark mode}
	I've included a simple way to use a dark mode
	color theme. To use the dark color mode, provide the \texttt{darkmode}
	option.
	\begin{verbatim}
    \usetheme[darkmode]{pureminimalistic}
    \end{verbatim}
	Sometimes, the logos have to be changed to look nice on a
	dark background. For now, I am simply loading different
	files if \texttt{darkmode} is used.
\end{frame}

\section{Graphics}
\begin{frame}{Logos}
	Commands setting the logos:
	\begin{itemize}
		\item \texttt{\textbackslash{}logotitle} -- Command used for the title page.
		      Here \texttt{\textbackslash{}linewidth} corresponds to the entire paper width.
		\item \texttt{\textbackslash{}logoheader} -- Command used for the header.
		      Here \texttt{\textbackslash{}linewidth} corresponds to a smaller box,
		      as the horizontal space is shared with the title.
		\item \texttt{\textbackslash{}logofooter} -- Command used for the footer.
		      Here \texttt{\textbackslash{}linewidth} corresponds to a smaller box,
		      as the horizontal space is shared with the footer text.
	\end{itemize}
\end{frame}

\begin{frame}[fragile]{Logos -- Load own logo}
	To use your own logos, simply redefine the commands and adjust the sizes.
	\begin{verbatim}
 \renewcommand{\logotitle}{\includegraphics%
   [width=.2\linewidth]{alternative_logo/gameboy.png}}
 \renewcommand{\logoheader}{\includegraphics%
   [width=.5\linewidth]{alternative_logo/gameboy.png}}
 \renewcommand{\logofooter}{\includegraphics%
   [width=.15\linewidth]{alternative_logo/console.png}}
  \end{verbatim}
\end{frame}

\begin{frame}[fragile]{Logos -- Disable logo}
	To disable the logo, overwrite the default logo command with an empty
	command.
	\begin{verbatim}
 \renewcommand{\logoheader}{}
  \end{verbatim}
	You may want to add some vertical space if you wish to delete the \texttt{logoheader}.
	\begin{verbatim}
 \renewcommand{\logoheader}{\vspace{1.5em}}
  \end{verbatim}
\end{frame}

\begin{frame}{Figures}
	I also changed the default caption settings to not
	include \texttt{Figure:} and reduced the font size.
	\begin{figure}[H]
		\centering
		\begin{columns}[T]
			\begin{column}{.3\linewidth}
				\includegraphics[width=\linewidth]{example-image-a}
				\caption{Example A}
			\end{column}
			\begin{column}{.3\linewidth}
				\includegraphics[width=\linewidth]{example-image-b}
				\caption{Example B}
			\end{column}
		\end{columns}
	\end{figure}
\end{frame}

\begin{frame}[fragile]{Figures -- Set background watermark}
	There is no extra option to define a background watermark, but here
	is a command that could be used to create one manually:
	\vfill
	\begin{verbatim}
\setbeamertemplate{background}{%
  \tikz[overlay,remember picture]%
  \node[opacity=0.8]at (current page.center)%
  {\includegraphics[width=.2\linewidth]%
  {example-image-a}};%
}
  \end{verbatim}
\end{frame}

{
\setbeamertemplate{background}{%
	\tikz[overlay,remember picture]%
	\node[opacity=0.8]at (current page.center)%
	{\includegraphics[width=.2\linewidth]%
		{example-image-a}};%
}
\begin{frame}{Figures -- Set background watermark}
	Usually you would add this command to specific
	frames by enclosing this command and all desired frames with
	curly brackets.
	\vfill
	See the source code of this \emph{*.tex} file for an
	example.
\end{frame}
}

\section{Footer options}
\begin{frame}[fragile]{Disable footer}
	If you do not want to use a footer, disable it with:
	\begin{verbatim}
    \usetheme[nofooter]{pureminimalistic}
  \end{verbatim}
\end{frame}

\begin{frame}[fragile]{Show max slide numbers}
	For these slides, I used the option to
	show the maximum number of slides. To activate it
	one has to activate it with:
	\begin{verbatim}
    \usetheme[showmaxslides]{pureminimalistic}
  \end{verbatim}
	Usually, I prefer to not show the maximum number of
	slides, as the people tend to lose focus if they know
	the last few slides are shown.
\end{frame}

\begin{frame}[fragile]{Remove footer logo}
	If you wish to remove the footer logo \emph{and}
	move the page number to the right parts use:
	\begin{verbatim}
    \usetheme[nofooterlogo]{pureminimalistic}
  \end{verbatim}
\end{frame}

\begin{frame}[fragile]{Change Page word}
	If you wish to remove or change the word \emph{Page}
	in the footer, change the value with
	\begin{verbatim}
    \renewcommand{\pageword}{Seite}
  \end{verbatim}
\end{frame}

\section{Citations}
\begin{frame}{Citations}
	I've also changed the bibliography options to be minimalistic:

	Just showing a simple \texttt{\textbackslash{}cite} \cite{AlexNet}
	\vfill
	\printbibliography
\end{frame}

\appendix % do not count the following slides for the total number
\section*{Backup Slides}
\begin{frame}[plain, noframenumbering]
	\centering
	\vfill
	{\fontsize{40}{50}\selectfont Backup Slides}
	\vfill
\end{frame}

\begin{frame}{What happened to the page numbering?}
	\begin{itemize}
		\item I've used the \texttt{appendixnumberbeamer}
		      package, which resets the frame counting after calling
		      \texttt{\textbackslash{}appendix}
		\item Depending on the used pdf viewer, the total
		      count of frames shouldn't include the backup slides and
		      won't demotivate the audience.
		\item Usually, I would use a \texttt{plain} frame
		      for the backup slides.
	\end{itemize}
\end{frame}

\end{document}
